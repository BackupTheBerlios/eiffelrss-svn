%===============================================================================
% SYNDICATION Developer Guide
%===============================================================================
% $Id$
%===============================================================================


%===============================================================================
% Configuration
%===============================================================================


%-------------------------------------------------------------------------------
% \documentclass and \usepackage directives
%-------------------------------------------------------------------------------
\documentclass[a4paper,fleqn]{report}
%\usepackage{ngerman}
\usepackage[latin1]{inputenc}
\usepackage[T1]{fontenc}
\usepackage[small,hang,bf]{caption2}
\usepackage{fancyhdr}
\usepackage[nice]{nicefrac}
\usepackage{color,listings}
\usepackage{alltt}


% Compilation with latex or pdflatex?
\newif\ifpdf 
\ifx\pdfoutput\undefined 
  \pdffalse
\else
  \pdfoutput=1 
  \pdftrue 
\fi 

% Compilation with pdflatex
\ifpdf
 
  \usepackage[pdftex]{graphicx}

  \usepackage[
    pdftex,
    a4paper,
    bookmarks,
    pdfstartview=FitH,    % starts with page width
    bookmarksopen,        % opens index
    bookmarksnumbered,    % index with numbering
    colorlinks,           % links with color, otherwise with border
    linkcolor=blue,       % Standard red
    citecolor=blue,       % Standard green
    urlcolor=magenta,     % Standard cyan
    filecolor=blue
  ]{hyperref} 

  \pdfinfo{
    /Title      (EiffelRSS SYNDICATION Developer Guide)
    /Author     (Thomas Weibel, Martin Luder, Michael K�ser)
    /Subject    (Eiffel programming)
    /Keywords   (Programming, EiffelRSS)
  }

  % Use default Acrobat reader fonts
  \usepackage{mathpazo}

  % Use CM fonts (increases document size)
  % \usepackage{ae}

% Compilation with latex
\else 

  \usepackage{graphicx} 

\fi


%-------------------------------------------------------------------------------
% Configure \maketitle
%-------------------------------------------------------------------------------
\title{EiffelRSS \\ SYNDICATION \\ Developer Guide}
\author{
  Michael K\"aser <kaeserm@student.ethz.ch>
  \and 
  Martin Luder <luderm@student.ethz.ch>
  \and 
  Thomas Weibel <weibelt@student.ethz.ch>
}
\date{\today}


%-------------------------------------------------------------------------------
% Configure fancyhdr
%-------------------------------------------------------------------------------
\pagestyle{fancy}

\renewcommand{\headrulewidth}{0.1 pt}
\renewcommand{\footrulewidth}{0.1 pt}

\fancypagestyle{plain}{
  \lhead{\nouppercase{\leftmark}}
  \chead{}
  \rhead{\thepage}
  \lfoot{EiffelRSS}
  \cfoot{}
  \rfoot{SYNDICATION Developer Guide}
}

\lhead{\nouppercase{\leftmark}}
\chead{}
\rhead{\thepage}

\lfoot{EiffelRSS}
\cfoot{}
\rfoot{SYNDICATION Developer Guide}


%-------------------------------------------------------------------------------
% Configure listings
%-------------------------------------------------------------------------------
\lstset{showstringspaces=false,
  breaklines=true,
  breakindent=0pt,
  prebreak=\mbox{\tiny$\searrow$},
  postbreak=\mbox{{\color{blue}\tiny$\rightarrow$}},
  frame=trBL,
  framerule=0.75pt,
  framesep=4pt,
  rulesep=0.75pt  
}


%-------------------------------------------------------------------------------
% Common configuration
%-------------------------------------------------------------------------------
\setlength{\parindent}{0em}
\setlength{\parskip}{1.5ex plus0.5ex minus0.5ex}
\sloppy
\setlength{\mathindent}{0em}


%-------------------------------------------------------------------------------
% Commandos
%-------------------------------------------------------------------------------
\newcommand{\hr}{\rule{\textwidth}{1pt}}


%===============================================================================
% Document
%===============================================================================
\begin{document}

\begin{titlepage}
  \newlength{\centeroffset}
  \setlength{\centeroffset}{-0.5\oddsidemargin}
  \addtolength{\centeroffset}{0.5\evensidemargin}

  \thispagestyle{empty}

  \noindent\includegraphics[width=\textwidth]{../../figures/big_ETH}\\[-3mm]
  \hr

  \vspace*{\stretch{1}}

  \makebox[0pt][l]{
    \begin{minipage}{\textwidth}
      \flushright{
        \Huge\bfseries EiffelRSS
      }

      \noindent\rule{\textwidth}{3pt}\\[2.5ex]

      \hfill\emph{
        \Large SYNDICATION Developer Guide
      }
    \end{minipage}
  }

  \vspace{\stretch{1}}

  \makebox[0pt][l]{
    \begin{minipage}{\textwidth}
      \flushright{
        \bfseries 
        Michael K\"aser <kaeserm@student.ethz.ch>\\[0.3ex]
        Martin Luder <luderm@student.ethz.ch>\\[0.3ex]
        Thomas Weibel <weibelt@student.ethz.ch>\\[0.3ex]
      }
    \end{minipage}
  }

  \vspace{\stretch{1}}

  \noindent\hr\\[1mm]
  \includegraphics[width=\textwidth]{../../figures/big_inf}
\end{titlepage}

% Use roman page numbering
\pagenumbering{roman}

\begin{abstract}
  \texttt{SYNDICATION} is the main cluster of EiffelRSS with a feed
  object model and classes to load / write feeds. It is divided into
  three subclusters.
\end{abstract}

\clearpage
\tableofcontents

\clearpage
\listoffigures

\newpage

% Set page counter to zero
\setcounter{page}{0} 

% Use arabic page numbering
\pagenumbering{arabic}


%-------------------------------------------------------------------------------
% Part: INTERFACE
%-------------------------------------------------------------------------------
\include{parts/interface}


%-------------------------------------------------------------------------------
% Part: FEED
%-------------------------------------------------------------------------------
\part{FEED}

\chapter{Overview}
\label{cha:feed-overview}

\texttt{FEED} is the central datastructure of EiffelRSS. It defines an
abstract syndication feed.

\begin{figure}[htbp]
  \centering
  \includegraphics[width=\textwidth]{./figures/EIFFELRSS_SYNDICATION_FEED}
  \caption{BON diagram of cluster \texttt{FEED}}
  \label{fig:formats}
\end{figure}


\chapter{Usage}
\label{cha:feed-usage}

\begin{lstlisting}[language=Eiffel]
class
  FEED_EXAMPLE

create
  make

feature -- Initialization

  make is
      -- Creation procedure.
    do
      -- Create a simple feed with some categories
      create feed.make ("EiffelRSS", create {HTTP_URL}.make ("http://eiffelrss.berlios.de"), "EiffelRSS news")
      feed.add_category (create {CATEGORY}.make_title ("RSS"))
      feed.add_category (create {CATEGORY}.make_title ("Programming"))
      feed.add_category (create {CATEGORY}.make_title ("Eiffel"))
      
      -- Add a cloud to feed
      feed.create_cloud ("eiffelrss.berlios.de", 80, "/RPC2", "xmlStorageSystem.rssPleaseNotify", "xml-rpc")
      
      -- Add an image to feed
      feed.create_image (create {HTTP_URL}.make ("http://eiffelrss.berlios.de/logo.png"), "EiffelRSS", create {HTTP_URL}.make ("http://eiffelrss.berlios.de"))
      
      -- Add a text input field to feed
      feed.create_text_input ("Search", "Search award-winning pages", "search", create {HTTP_URL}.make ("http://eiffelrss.berlios.de/Main/SearchWiki/"))
      
      -- Add some simple items, use `feed.last_added_item' or directly create an item for finer control
      feed.new_item ("Version 23 released!", create {HTTP_URL}.make ("http://eiffelrss.berlios.de/Main/News"), "Version 23 of EiffelRSS got release today. Happy syndicating!")
      feed.last_added_item.add_category (create {CATEGORY}.make_title_domain ("News", create {HTTP_URL}.make ("http://eiffelrss.berlios.de/Main/News/")))
      
      feed.new_item ("EiffelRSS wins award", create {HTTP_URL}.make ("http://eiffelrss.berlios.de/Main/Awards"), "EiffelRSS has been awarded by ISE as best syndication software written in Eiffel. For more info see award-winning pages: http://eiffelrss.berlios.de")
      feed.last_added_item.set_guid (create {ITEM_GUID}.make_perma_link ("http://eiffelrss.berlios.de/newsItem42"))
        
      -- Print feed
      io.put_string ("Sample feed:%N")
      io.put_string ("============%N%N%N")
      io.put_string (feed.to_string)
    end
    
feature -- Arguments

  feed: FEED
      -- Example feed

end -- class FEED_EXAMPLE
\end{lstlisting}


\chapter{Feed implementation}
\label{cha:feed-implementation}


\section{Class FEED}
\label{sec:feed-feed}


\section{Class CHANNEL}
\label{sec:feed-channel}


\section{Class ITEM}
\label{sec:feed-item}


\section{Class CATEGORY}
\label{sec:feed-category}


\section{Observers}
\label{sec:feed-observers}



%-------------------------------------------------------------------------------
% Part: FORMATS
%-------------------------------------------------------------------------------
\include{parts/formats}

\end{document}
