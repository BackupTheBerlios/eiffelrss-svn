%===============================================================================
% Presentation
%===============================================================================
% $Id: presentation.tex 159 2005-01-24 18:31:36Z thom $
%===============================================================================


%===============================================================================
% Configuration
%===============================================================================


%-------------------------------------------------------------------------------
% \documentclass and \usepackage directives
%-------------------------------------------------------------------------------
\documentclass{beamer}
\usepackage{beamerthemebars}
\usepackage[english]{babel}
\usepackage[latin1]{inputenc}


%-------------------------------------------------------------------------------
% Use some nice templates
%-------------------------------------------------------------------------------
\beamertemplateshadingbackground{blue!10}{structure!10}
\beamertemplatetransparentcovereddynamic
\beamertemplateballitem
\beamertemplatenumberedballsectiontoc
\colorlet{beameralert}{blue!60!black}


%-------------------------------------------------------------------------------
% Configure \titlepage
%-------------------------------------------------------------------------------
\title{EiffelRSS}
\author{Martin Luder \and Michael K�ser \and Thomas Weibel}


%===============================================================================
% Document
%===============================================================================
\begin{document}


% Titlepage
\frame{
  \titlepage
}

\section{Overview}

\frame{
  \frametitle{What is EiffelRSS?}

  \begin{itemize}[<+-| alert@+>]
  \item EiffelRSS is an Eiffel library to parse RSS. The goal is to
    provide the Eiffel development community with an easy to use and
    well structured API for RSS.
  \item The distribution also contains a RSS newsfeed reader
    written with EiffelVision and EiffelRSS.
  \end{itemize}
}


\section{RSS}

\frame{
  \frametitle{What is RSS?}

  \begin{itemize}[<+-| alert@+>]
  \item Foo
  \item Bar
  \item SNAFU
  \item 23 Skidoo
  \end{itemize}
}


\section{Project Management}

\frame{
  \frametitle{Project Management}

  \begin{itemize}[<+-| alert@+>]
  \item developer.berlios.de
  \item Subversion
  \item PmWiki
  \item Gobo Eiffel test (getest)
  \end{itemize}
}


\section{Problems}

\frame{
  \frametitle{Problems}

  \begin{itemize}[<+-| alert@+>]
  \item Foo
  \item Bar
  \item SNAFU
  \item 23 Skidoo
  \end{itemize}
}


\section{Statistics}

\frame{
  \frametitle{Statistics}

  \begin{itemize}[<+-| alert@+>]
  \item Number of Eiffel code lines: A + B + C
    \begin{itemize}
    \item Library: A
    \item Newsreader: B
    \item Examples and testing framework: C
    \end{itemize}
  \item Number of Eiffel classes: X + Y + Z
    \begin{itemize}
    \item Library: X
    \item Newsreader: Y
    \item Examples and testing framework: Z
    \end{itemize}
  \end{itemize}
}


\section{Library}

\frame{
  \frametitle{EiffelRSS library}

  [BON diagram]
}


\section{Newsreader}

\frame{
  \frametitle{EiffelRSS newsreader}

  [Screenshot, Demo]
}


\section{Download}

\frame{
  \frametitle{Getting EiffelRSS}

  \begin{itemize}[<+-| alert@+>]
  \item Subversion
    \begin{itemize}
    \item svn checkout ...
    \end{itemize}
  \item Archive
    \begin{itemize}
    \item http://eiffelrss.berlios.de
    \end{itemize}
  \end{itemize}
}

\end{document}
